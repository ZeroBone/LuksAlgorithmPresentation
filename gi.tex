% \documentclass[handout]{beamer}
\documentclass{beamer}
\usepackage{lmodern}
\usepackage[utf8]{inputenc}
\usepackage[english]{babel}

\usepackage{amsmath, amsthm, amsfonts, amssymb}
\usepackage{mathrsfs}

\usepackage{oubraces}
\usepackage{array}

\usepackage{tikz}
\usetikzlibrary{calc, intersections, positioning}

\usepackage[sanserif]{complexity}

% Algorithms
\usepackage[ruled]{algorithm2e}
\SetKw{Continue}{continue}

\usetheme{Madrid}
\usecolortheme{default}

\title[On the graph isomorphism problem]{On the graph isomorphism problem}
\author{Alexander Mayorov}
\institute[MPI-INF]{Max-Planck Institute for Informatics}

\newlang{\SI}{SI}
\renewclass{\EXP}{EXPTIME}

\def\N{\mathbb{N}_{\ge 0}}
\def\Npos{\mathbb{N}_{\ge 1}}
\def\Nsimpl{\mathbb{N}}
\def\R{\mathbb{R}}
\def\Z{\mathbb{Z}}
\def\Q{\mathbb{Q}}

\def\X{\mathcal{X}}

\DeclareMathOperator{\Sym}{Sym}
\DeclareMathOperator{\Iso}{Iso}

% expected value
\def\ExpVal{\E}

\newcommand{\var}[1]{\textcolor{blue}{#1}}
\newcommand{\sol}[1]{\textcolor{orange}{#1}}

\definecolor{OliveGreen}{rgb}{0,0.6,0}
\definecolor{LightCyan}{rgb}{0.88,1,1}
\definecolor{MagentaZB}{RGB}{255,0,255}
\definecolor{LightBlueZB}{RGB}{51,51,179} % #3333B3

\newenvironment<>{openproblem}[1][]{%
	\setbeamercolor{block title}{fg=white,bg=magentazb}%
	\begin{block}#1}{\end{block}}

\begin{document}
	
\frame{\titlepage}

\begin{frame}{Graph Isomorphism Problem}
	
	% \textbf{Intuition}: Two graphs are called isomorphic whenever they are equal up to renaming of vertices.
	
	\begin{definition}[Graph isomorphism]
		Undirected graphs \( G_1 = (V_1, E_1) \) and \( G_2 = (V_2, E_2) \) are called \textbf{isomorphic} (denoted $ G_1 \cong G_2 $) if there exists a bijection $ f : V_1 \to V_2 $ such that \[
		\{u,v\} \in E_1 \iff \{f(u), f(v)\} \in E_2.
		\]
	\end{definition}

\newcommand{\graphA}{%
	\raisebox{-.5\height}{%
		\begin{tikzpicture}[every node/.style={circle, draw, fill=gray!10, inner sep=1pt, minimum size=18pt}, scale=0.8]
			\node (a1) at (0,0) {1};
			\node (a2) at (1.2,0) {2};
			\node (a3) at (1.2,1.2) {3};
			\node (a4) at (0,1.2) {4};
			\draw (a1)--(a2)--(a3)--(a4)--(a1);
		\end{tikzpicture}%
	}%
}

\newcommand{\graphB}{%
	\raisebox{-.5\height}{%
		\begin{tikzpicture}[every node/.style={circle, draw, fill=gray!10, inner sep=1pt, minimum size=18pt}, scale=0.8]
			\node (b1) at (0,0) {a};
			\node (b2) at (1.2,0) {b};
			\node (b3) at (1.2,1.2) {c};
			\node (b4) at (0,1.2) {d};
			\draw (b1)--(b3);
			\draw (b3)--(b4);
			\draw (b4)--(b2);
			\draw (b2)--(b1);
		\end{tikzpicture}%
	}%
}

\newcommand{\graphC}{%
	\raisebox{-.5\height}{%
		\begin{tikzpicture}[every node/.style={circle, draw, fill=gray!10, inner sep=1pt, minimum size=18pt}, scale=0.8]
			\node (c1) at (0,0) {$ \alpha $};
			\node (c2) at (1.2,0) {$ \beta $};
			\node (c3) at (2.4,0) {$ \gamma $};
			\node (c4) at (1.2,1.2) {$ \delta $};
			\draw (c1)--(c2)--(c3);
			\draw (c2)--(c4);
		\end{tikzpicture}%
	}%
}

\begin{example}
	\begin{center}
		\renewcommand{\arraystretch}{1.5} % row height
		\begin{tabular}{c c c c c}
			\graphA & $\vphantom{\graphA}\cong$ & \graphB & $\vphantom{\graphA}\not\cong$ & \graphC
		\end{tabular}
	\end{center}
\end{example}

\begin{definition}[Graph Isomorphism Problem ($ \GI $)]
	\textbf{Given}: Undirected graphs $ G_1 $, $ G_2 $
	
	\textbf{Decide}: Is $ G_1 \cong G_2 $?
\end{definition}
\end{frame}

\begin{frame}{$ \GI $ has a very special status (for all we know)}
	\begin{center}
		\begin{tikzpicture}[scale=1.0, every node/.style={font=\large}]
			% Parameters
			\def\semiMajorAxis{5.5}
			\def\semiMinorAxis{3.5}
			\def\npSubclassLabelOffset{-1.5}
			\def\npSubclassExampleOffset{-1}
			\def\pYCoord{-2.0}
			\def\npiYCoord{0.0}
			\def\npcYCoord{2.0}
			
			\path[use as bounding box] (-6,-3.2) rectangle (6,3.2);
			
			% NP ellipse (main region)
			\path[name path=ellipse] (0,0) ellipse ({\semiMajorAxis} and {\semiMinorAxis});
			\draw[fill=blue!10, thick, name path global=NP] (0,0) ellipse ({\semiMajorAxis} and {\semiMinorAxis});
			\node at (4.7,2.6) {\textbf{NP}};
			
			% Define y positions of boundaries
			\def\yP{-1.0}
			\def\yC{1.0}
			
			% Define horizontal lines for boundaries
			\path[name path=Pline] (-\semiMajorAxis-1,\yP) -- (\semiMajorAxis+1,\yP);
			\path[name path=Cline] (-\semiMajorAxis-1,\yC) -- (\semiMajorAxis+1,\yC);
			
			% Find intersection points
			\path[name intersections={of=ellipse and Pline, by={PL, PR}}];
			\path[name intersections={of=ellipse and Cline, by={CL, CR}}];
			
			% Draw boundary lines exactly clipped to ellipse
			\draw[dashed, thick] (PL) -- (PR);
			\draw[dashed, thick] (CL) -- (CR);
			
			% Region labels
			\node[font=\bfseries, OliveGreen!80!black, anchor=east] at (\npSubclassLabelOffset,\pYCoord) {P};
			\node[font=\bfseries, anchor=east] at (\npSubclassLabelOffset,\npiYCoord) {NP-intermediate};
			\node[font=\bfseries, red!80!black, anchor=east] at (\npSubclassLabelOffset,\npcYCoord) {NP-complete};
			
			% Example problems (directly inside regions)
			% NPC
			\node[anchor=west] at (\npSubclassExampleOffset,\npcYCoord + 0.5) {SAT, Clique, Coloring,};
			\node[anchor=west] at (\npSubclassExampleOffset,\npcYCoord) {Vertex-Cover, Set-Cover,};
			\node[anchor=west] at (\npSubclassExampleOffset,\npcYCoord - 0.5) {Hitting-Set, Subset-Sum, $ \dots $};
			% NPI
			\node[MagentaZB, anchor=west] (ginode) at (\npSubclassExampleOffset,\npiYCoord) {Graph Isomorphism ($ \GI $)};
			\node[anchor=west] at ($ (ginode.east |- 0,\npiYCoord) + (-0.25,-0.1) $) {, $\dots$};
			% P
			\node[anchor=west] at (\npSubclassExampleOffset,\pYCoord + 0.5) {2-SAT, Edge-Cover, 2-Coloring,};
			\node[anchor=west] at (\npSubclassExampleOffset,\pYCoord) {Primes, Maximal-Matching,};
			\node[anchor=west] at (\npSubclassExampleOffset,\pYCoord - 0.5) {Shortest-Path, $ \dots $};
			
		\end{tikzpicture}
	\end{center}
	\textbf{Warning}: this diagram assumes $ \PH $ does not collapse (long-standing open question) and $ \GI \notin \P $ (open).
\end{frame}

\begin{frame}{Why is $ \GI $ (very likely) $ \NP $-intermediate?}
	Intuitively: \begin{itemize}
		\item $ \NP $-complete problems have purely ``search'' flavor
		\item $ \GI $ is a problem of ``comparison'' flavor
	\end{itemize}

	The ``comparison'' flavor of $ \GI $ can be rigorously captured as follows:
	
	\begin{theorem}
		$ \overline{\GI} \in \AM $
	\end{theorem}

	\begin{corollary}
		$ \GI $ is not $ \NP $-complete unless $ \PH = \Sigma_2 \P $.
	\end{corollary}
	\begin{proof}
		$ \NP $-hardness of $ \GI $ implies $ \coNP \subseteq \AM $ by above theorem. Since $ \AM \subseteq \Pi_2 \P $ we have $ \Sigma_2 \P = \Pi_2 \P $ and thus $ \PH = \Sigma_2 \P $.
	\end{proof}
\end{frame}

\begin{frame}{High-level overview}
	content...
\end{frame}

\begin{frame}{Strings and permutations}
	% Fix some finite set $ \Omega $ and a finite alphabet $ \Sigma $.
	
	A string is a mapping $ u : \Omega \rightarrow \Sigma $ where $ \Omega, \Sigma $ are fixed finite sets.
	
	\textbf{Intuition}: Elements of $ \Omega $ are markers of positions, and values of $ u $ are characters at the corresponding positions.
	
	\begin{definition}[Permutation applied to a string]
		For $ \sigma \in \Sym(\Omega) $, define string $ u^\sigma $ to be \begin{align*}
			u^\sigma : \Omega &\rightarrow \Sigma \\
			\omega &\mapsto u (\sigma^{-1}(\omega))
		\end{align*}
	\end{definition}

	\begin{example}
		Let $ \Omega := \Sym(\{1, 2, 3\}) =: S_3 $ and $ \Sigma := \{a, b, c\} $. Then \[
			abc^{()} = abc \qquad abc^{(1, 2, 3)} = cab \qquad abc^{(1, 3, 2)} = bca
		\] where $ xyz \in \Sigma^3 $ denotes mapping $ 1 \mapsto x $, $ 2 \mapsto y $, $ 3 \mapsto z $.
	\end{example}

\end{frame}

\begin{frame}{String Isomorphism Problem}
	
	Let $ u, v $ be strings and $ \Gamma \le \Sym(\Omega) $. Define \[
		\Iso_\Gamma(u, v) := \{\gamma \in \Gamma \mid u^\gamma = v\}
	\] We say that $ u $ is $ \Gamma $-isomorphic to $ v $ and write $ u \cong_\Gamma v $ whenever $ \Iso_\Gamma(u, v) \neq \varnothing $.
	
	\begin{definition}[String Isomorphism Problem ($ \SI $)]
		\textbf{Given}: \begin{itemize}
			\item Strings $ u, v $
			\item A generating set of a group $ \Gamma \le \Sym(\Omega) $
		\end{itemize}
	
		\textbf{Decide}: Is $ u \cong_\Gamma v $?
	\end{definition}
\end{frame}
	
\end{document}