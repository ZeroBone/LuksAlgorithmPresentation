\documentclass[handout]{beamer}
%\documentclass{beamer}
\usepackage{lmodern}
\usepackage[utf8]{inputenc}
\usepackage[english]{babel}

\usepackage{amsmath, amsthm, amsfonts, amssymb}
\usepackage{mathrsfs}

\usepackage{oubraces}
\usepackage{array}
\usepackage{varwidth}
\usepackage{array}

\usepackage{tikz}
\usetikzlibrary{intersections}
\usetikzlibrary{calc}
\usetikzlibrary{patterns}
\usetikzlibrary{chains}
\usetikzlibrary{decorations.markings}
\usetikzlibrary{decorations.pathreplacing}
\usetikzlibrary{positioning}
\usetikzlibrary{shadows}
\usetikzlibrary{shapes.arrows}
\usetikzlibrary{hobby}

\tikzset{%
	decision/.style = {diamond,draw, fill=blue!50},
	line/.style = {draw, -stealth, thick},
	block/.style = {rectangle, draw, minimum height=10mm,
		align=center}
}

\tikzset{suspend join/.code={\def\tikz@after@path{}}}

\tikzstyle{shapeaftershift}=[thick,dashed]
\tikzstyle{shapebeforeshift}=[gray,opacity=.5]

\newcommand*{\blocktext}[1]{%
	\begin{varwidth}{9em}%
		\centering
		#1%
	\end{varwidth}%
}

\usepackage[sanserif]{complexity}

% Algorithms
\usepackage[ruled]{algorithm2e}
\SetKw{Continue}{continue}

\usetheme{Madrid}
\usecolortheme{default}

\title[On the Graph Isomorphism problem]{On the Graph Isomorphism problem}
\author{Alexander Mayorov}
\institute[MPI-INF]{Max-Planck Institute for Informatics}

\newlang{\SI}{SI}
\renewclass{\EXP}{EXPTIME}

\def\N{\mathbb{N}_{\ge 0}}
\def\Npos{\mathbb{N}_{\ge 1}}
\def\Nsimpl{\mathbb{N}}
\def\R{\mathbb{R}}
\def\Z{\mathbb{Z}}
\def\Q{\mathbb{Q}}

\def\X{\mathcal{X}}

\DeclareMathOperator{\Sym}{Sym}
\DeclareMathOperator{\Iso}{Iso}
\DeclareMathOperator{\ImageOf}{Im}
\DeclareMathOperator{\KernelOf}{Ker}
\DeclareMathOperator{\Stab}{Stab}
\DeclareMathOperator{\Aut}{Aut}

\newcommand{\id}{\mathrm{id}}
\newcommand{\sign}{\mathrm{sign}}

% expected value
\def\ExpVal{\E}

\newcommand{\gen}[1]{\langle #1 \rangle}
\newcommand{\restrict}[1]{\raisebox{-.5ex}{$|$}_{#1}}
\newcommand{\abs}[1]{\left\vert#1\right\vert}

\definecolor{OliveGreenZB}{rgb}{0,0.6,0}
\definecolor{DarkRedZB}{rgb}{0.8,0,0}
\definecolor{LightCyanZB}{rgb}{0.88,1,1}
\definecolor{MagentaZB}{RGB}{255,0,255}
\definecolor{LightBlueZB}{RGB}{51,51,179} % #3333B3
\definecolor{DarkOrangeZB}{RGB}{204,102,0}
\definecolor{TurquoiseZB}{RGB}{87,180,186}

\newtheorem{proposition}[theorem]{Proposition}

\newenvironment<>{openproblem}[1][]{%
	\setbeamercolor{block title}{fg=white,bg=magentazb}%
	\begin{block}#1}{\end{block}}

\begin{document}
	
\frame{\titlepage}

\begin{frame}{Graph Isomorphism Problem}
	
	% \textbf{Intuition}: Two graphs are called isomorphic whenever they are equal up to renaming of vertices.
	
	\begin{definition}[Graph isomorphism]
		Undirected graphs \( G_0 = (V_0, E_0) \) and \( G_1 = (V_1, E_1) \) are called \textbf{isomorphic} (denoted $ G_0 \cong G_1 $) if there exists a bijection $ \sigma : V_0 \to V_1 $ such that \[
		\{u,v\} \in E_0 \iff \{\sigma(u), \sigma(v)\} \in E_1
		\]
	\end{definition}

\newcommand{\graphA}{%
	\raisebox{-.5\height}{%
		\begin{tikzpicture}[every node/.style={circle, draw, fill=gray!10, inner sep=1pt, minimum size=18pt}, scale=0.8]
			\node (a1) at (0,0) {1};
			\node (a2) at (1.2,0) {2};
			\node (a3) at (1.2,1.2) {3};
			\node (a4) at (0,1.2) {4};
			\draw (a1)--(a2)--(a3)--(a4)--(a1);
		\end{tikzpicture}%
	}%
}

\newcommand{\graphB}{%
	\raisebox{-.5\height}{%
		\begin{tikzpicture}[every node/.style={circle, draw, fill=gray!10, inner sep=1pt, minimum size=18pt}, scale=0.8]
			\node (b1) at (0,0) {a};
			\node (b2) at (1.2,0) {b};
			\node (b3) at (1.2,1.2) {c};
			\node (b4) at (0,1.2) {d};
			\draw (b1)--(b3);
			\draw (b3)--(b4);
			\draw (b4)--(b2);
			\draw (b2)--(b1);
		\end{tikzpicture}%
	}%
}

\newcommand{\graphC}{%
	\raisebox{-.5\height}{%
		\begin{tikzpicture}[every node/.style={circle, draw, fill=gray!10, inner sep=1pt, minimum size=18pt}, scale=0.8]
			\node (c1) at (0,0) {$ \alpha $};
			\node (c2) at (1.2,0) {$ \beta $};
			\node (c3) at (2.4,0) {$ \gamma $};
			\node (c4) at (1.2,1.2) {$ \delta $};
			\draw (c1)--(c2)--(c3);
			\draw (c2)--(c4);
		\end{tikzpicture}%
	}%
}

\begin{example}
	\begin{center}
		\renewcommand{\arraystretch}{1.5} % row height
		\begin{tabular}{c c c c c}
			\graphA & $\cong$ & \graphB & $\not\cong$ & \graphC
		\end{tabular}
	\end{center}
\end{example}

\begin{definition}[Graph Isomorphism Problem ($ \GI $)]
	\textbf{Given}: Undirected graphs $ G_0 = (V_0, E_0) $, $ G_1 = (V_1, E_1) $
	
	\textbf{Decide}: Is $ G_0 \cong G_1 $?
\end{definition}
\end{frame}

\begin{frame}{$ \GI $ has a very special status (for all we know)}
	\begin{center}
		\begin{tikzpicture}[scale=1.0, every node/.style={font=\large}]
			% Parameters
			\def\semiMajorAxis{5.5}
			\def\semiMinorAxis{3.5}
			\def\npSubclassLabelOffset{-1.5}
			\def\npSubclassExampleOffset{-1}
			\def\pYCoord{-2.0}
			\def\npiYCoord{0.0}
			\def\npcYCoord{2.0}
			
			\path[use as bounding box] (-6,-3.2) rectangle (6,3.2);
			
			% NP ellipse (main region)
			\path[name path=ellipse] (0,0) ellipse ({\semiMajorAxis} and {\semiMinorAxis});
			\draw[fill=blue!10, thick, name path global=NP] (0,0) ellipse ({\semiMajorAxis} and {\semiMinorAxis});
			\node at (4.7,2.6) {\textbf{NP}};
			
			% Define y positions of boundaries
			\def\yP{-1.0}
			\def\yC{1.0}
			
			% Define horizontal lines for boundaries
			\path[name path=Pline] (-\semiMajorAxis-1,\yP) -- (\semiMajorAxis+1,\yP);
			\path[name path=Cline] (-\semiMajorAxis-1,\yC) -- (\semiMajorAxis+1,\yC);
			
			% Find intersection points
			\path[name intersections={of=ellipse and Pline, by={PL, PR}}];
			\path[name intersections={of=ellipse and Cline, by={CL, CR}}];
			
			% Draw boundary lines exactly clipped to ellipse
			\draw[dashed, thick] (PL) -- (PR);
			\draw[dashed, thick] (CL) -- (CR);
			
			% Region labels
			\node[font=\bfseries, OliveGreenZB!80!black, anchor=east] at (\npSubclassLabelOffset,\pYCoord) {P};
			\node[font=\bfseries, anchor=east] at (\npSubclassLabelOffset,\npiYCoord) {NP-intermediate};
			\node[font=\bfseries, DarkRedZB, anchor=east] at (\npSubclassLabelOffset,\npcYCoord) {NP-complete};
			
			% Example problems (directly inside regions)
			% NPC
			\node[anchor=west] at (\npSubclassExampleOffset,\npcYCoord + 0.5) {SAT, Clique, Coloring,};
			\node[anchor=west] at (\npSubclassExampleOffset,\npcYCoord) {Vertex-Cover, Set-Cover,};
			\node[anchor=west] at (\npSubclassExampleOffset,\npcYCoord - 0.5) {Hitting-Set, Subset-Sum, $ \dots $};
			% NPI
			\node[MagentaZB, anchor=west] (ginode) at (\npSubclassExampleOffset,\npiYCoord) {Graph Isomorphism ($ \GI $)};
			\node[anchor=west] at ($ (ginode.east |- 0,\npiYCoord) + (-0.25,-0.1) $) {, $\dots$};
			% P
			\node[anchor=west] at (\npSubclassExampleOffset,\pYCoord + 0.5) {2-SAT, Edge-Cover, 2-Coloring,};
			\node[anchor=west] at (\npSubclassExampleOffset,\pYCoord) {Primes, Maximal-Matching,};
			\node[anchor=west] at (\npSubclassExampleOffset,\pYCoord - 0.5) {Shortest-Path, $ \dots $};
			
		\end{tikzpicture}
	\end{center}
	\textbf{Warning}: this diagram assumes $ \PH $ does not collapse (long-standing open question) and $ \GI \notin \P $ (open).
\end{frame}

\begin{frame}{Why is $ \GI $ (very likely) $ \NP $-intermediate?}
	Intuitively: \begin{itemize}
		\item $ \NP $-complete problems have purely ``search'' flavor
		\item $ \GI $ is a problem of ``comparison'' flavor
	\end{itemize}

	The ``comparison'' flavor of $ \GI $ can be rigorously captured as follows:
	
	\begin{theorem}
		$ \GI \in \coAM $
	\end{theorem}

	\begin{corollary}
		$ \GI $ is not $ \NP $-complete unless $ \PH = \Sigma_2 \P $.
	\end{corollary}
	\begin{proof}
		$ \NP $-hardness of $ \GI $ implies $ \coNP \subseteq \AM $ by above theorem. Since $ \AM \subseteq \Pi_2 \P $ we have $ \Sigma_2 \P = \Pi_2 \P $ and thus $ \PH = \Sigma_2 \P $.
	\end{proof}
\end{frame}

\begin{frame}{High-level overview}
	We will show how to solve $ \GI $ via an intermediate problem:
	\begin{center}
		\begin{tikzpicture}[thick]
			% problems
			\node [block] (gi) {\blocktext{Graph Isomorphism Problem (GI)}};
			\node [block, right=4em of gi] (si) {\blocktext{String Isomorphism Problem (SI)}};
			
			% algorithms
			\node [below=1.5em of si] (luks) {Luks's algorithm};
			
			% reductions
			\draw[->, LightBlueZB] (gi) --node[midway, above]{reduce} (si);
			\draw[->, LightBlueZB] (si) --node[midway, right]{solve} (luks);
		\end{tikzpicture}
	\end{center}

	Motivation for introducing $ \SI $: \begin{itemize}
		\item In $ \GI $ we have to deal with \textbf{vertices} and \textbf{edges}
		\item In $ \SI $ we have to deal with only \textbf{characters}
		\item $ \rightsquigarrow $ $ \SI $ is much more convenient to work with
	\end{itemize}
\end{frame}

\begin{frame}{Strings and permutations}
	% Fix some finite set $ \Omega $ and a finite alphabet $ \Sigma $.
	
	A string is a mapping $ u : \Omega \rightarrow \Sigma $ where $ \Omega, \Sigma $ are fixed finite sets.
	
	\textbf{Intuition}: Elements of $ \Omega $ are markers of positions, and values of $ u $ are characters at the corresponding positions.
	
	\begin{definition}[Permutation applied to a string]
		For $ \sigma \in \Sym(\Omega) $, define string $ u^\sigma $ to be \begin{align*}
			u^\sigma : \Omega &\rightarrow \Sigma \\
			\omega &\mapsto u (\sigma^{-1}(\omega))
		\end{align*}
	\end{definition}

	\begin{example}
		Let $ \Omega := \{1, 2, 3\} $ and $ \Sigma := \{a, b, c\} $. Then \[
			abc^{()} = abc \qquad abc^{(1, 2, 3)} = cab \qquad abc^{(1, 3, 2)} = bca
		\] where $ xyz \in \Sigma^3 $ denotes mapping $ 1 \mapsto x $, $ 2 \mapsto y $, $ 3 \mapsto z $.
	\end{example}

\end{frame}

\begin{frame}{String Isomorphism Problem}
	
	Let $ u, v $ be strings and $ \Gamma \le \Sym(\Omega) $. Define \[
		\Iso_\Gamma(u, v) := \{\gamma \in \Gamma \mid u^\gamma = v\}
	\] We say that $ u $ is $ \Gamma $-isomorphic to $ v $ and write $ u \cong_\Gamma v $ whenever $ \Iso_\Gamma(u, v) \neq \varnothing $.
	
	\begin{definition}[String Isomorphism Problem ($ \SI $)]
		\textbf{Given}: \begin{itemize}
			\item Strings $ u, v $
			\item A generating set of a group $ \Gamma \le \Sym(\Omega) $
		\end{itemize}
	
		\textbf{Decide}: Is $ u \cong_\Gamma v $?
	\end{definition}

	\begin{example}
		Consider $ \Gamma := \gen{(1,\dots,n)} \le S_n = \Sym(\Omega) $ for $ \Omega := \{1, \dots, n\} $.
		
		Then $ u \cong_\Gamma v $ if and only if $ u $ and $ v $ are conjugate words.
	\end{example}
\end{frame}

\newcommand{\giToSiReductionGraphZero}{
	\raisebox{-.5\height}{
		\begin{tikzpicture}[every node/.style={circle, draw, fill=gray!10, inner sep=1pt, minimum size=18pt}, scale=0.8]
			\node (a3) at (0,0) {3};
			\node (a4) at (1.2,0) {4};
			\node (a2) at (1.2,1.2) {2};
			\node (a1) at (0,1.2) {1};
			\draw (a3)--(a1)--(a2)--(a3)--(a4);
		\end{tikzpicture}
	}
}

\newcommand{\giToSiReductionGraphOne}{
	\raisebox{-.5\height}{
		\begin{tikzpicture}[every node/.style={circle, draw, fill=gray!10, inner sep=1pt, minimum size=18pt}, scale=0.8]
			\node (b3) at (0,0) {1};
			\node (b4) at (1.2,0) {2};
			\node (b2) at (1.2,1.2) {4};
			\node (b1) at (0,1.2) {3};
			\draw (b3)--(b1)--(b2)--(b3)--(b4);
		\end{tikzpicture}
	}
}

% The essence of the above reduction is to ``lift`` isomorphism test from the level of nodes to the level of edges.

\begin{frame}{Towards a reduction from $ \GI $ to $ \SI $}
	\only<1|handout:1>{We reduce the following $ \GI $ instance to $ \SI $:\begin{center}
		\renewcommand{\arraystretch}{1.5} % row height
		\begin{tabular}{c c c}
			$ G_0 $ && $ G_1 $\\
			\giToSiReductionGraphZero & $\overset{?}{\cong}$ & \giToSiReductionGraphOne
		\end{tabular}
	\end{center}}
	\only<1|handout:1>{\textbf{Idea}: encode existence of edges into a string:}
	\only<1-2|handout:1-2>{\begin{center}
		\begin{tabular}{c|c c c c c c}
			$ \omega \in \Omega $ & $ \{1, 2\} $ & $ \{1, 3\} $ & $ \{1, 4\} $ & $ \{2, 3\} $ & $ \{2, 4\} $ & $ \{3, 4\} $\\
			\hline
			$ u_0(\omega) $ & $ 1 $ & $ 1 $ & $ 0 $ & $ 1 $ & $ 0 $ & $ 1 $\\
			$ u_1(\omega) $ & $ 1 $ & $ 1 $ & $ 1 $ & $ 0 $ & $ 0 $ & $ 1 $
		\end{tabular}
	\end{center}}
	\only<2-3|handout:2-3>{This $ \SI $ instance has the following solution:\begin{center}
		\begin{tikzpicture}[thick]
			
			% labels
			\node (l12) {$\{1,2\}$};
			\node[right=.5em of l12] (l13) {$\{1,3\}$};
			\node[right=.5em of l13] (l14) {$\{1,4\}$};
			\node[right=.5em of l14] (l23) {$\{2,3\}$};
			\node[right=.5em of l23] (l24) {$\{2,4\}$};
			\node[right=.5em of l24] (l34) {$\{3,4\}$};
			
			% u_0
			\node[below=.5em of l12] (u0at12) {$ 1 $};
			\node[below=.5em of l13] (u0at13) {$ 1 $};
			\node[below=.5em of l14] (u0at14) {$ 0 $};
			\node[below=.5em of l23] (u0at23) {$ 1 $};
			\node[below=.5em of l24] (u0at24) {$ 0 $};
			\node[below=.5em of l34] (u0at34) {$ 1 $};
			
			\node[left=2em of u0at12] {$ u_0 $};
			
			% v
			\node[below=3em of u0at12] (vat12) {$ 0 $};
			\node[below=3em of u0at13] (vat13) {$ 1 $};
			\node[below=3em of u0at14] (vat14) {$ 1 $};
			\node[below=3em of u0at23] (vat23) {$ 1 $};
			\node[below=3em of u0at24] (vat24) {$ 0 $};
			\node[below=3em of u0at34] (vat34) {$ 1 $};
			
			\node[left=2em of vat12] {$ v $};
			
			% edges from u_0 to v
			\draw[->, LightBlueZB] (u0at12.south) -- (vat14.north);
			\draw[->, LightBlueZB] (u0at13.south) -- (vat13.north);
			\draw[->, LightBlueZB] (u0at14.south) -- (vat12.north);
			\draw[->, LightBlueZB] (u0at23.south) -- (vat34.north);
			\draw[->, LightBlueZB] (u0at24.south) -- (vat24.north);
			\draw[->, LightBlueZB] (u0at34.south) -- (vat23.north);
			
			% u_1
			\node[below=3em of vat12] (u1at12) {$ 1 $};
			\node[below=3em of vat13] (u1at13) {$ 1 $};
			\node[below=3em of vat14] (u1at14) {$ 1 $};
			\node[below=3em of vat23] (u1at23) {$ 0 $};
			\node[below=3em of vat24] (u1at24) {$ 0 $};
			\node[below=3em of vat34] (u1at34) {$ 1 $};
			
			\node[left=2em of u1at12] {$ u_1 $};
			
			% edges from v to u_1
			\draw[->, OliveGreenZB] (vat12.south) -- (u1at23.north);
			\draw[->, OliveGreenZB] (vat13.south) -- (u1at13.north);
			\draw[->, OliveGreenZB] (vat14.south) -- (u1at34.north);
			\draw[->, OliveGreenZB] (vat23.south) -- (u1at12.north);
			\draw[->, OliveGreenZB] (vat24.south) -- (u1at24.north);
			\draw[->, OliveGreenZB] (vat34.south) -- (u1at14.north);
			
		\end{tikzpicture}
	\end{center}}
	\only<3|handout:3>{This solution shows the existence of the isomorphism\begin{center}
			\renewcommand{\arraystretch}{1.5} % row height
			\begin{tabular}{c c c}
				\giToSiReductionGraphZero & \begin{tikzpicture}[thick]
					\draw[->] (0,0) --node[midway, above]{$ \sigma := \textcolor{LightBlueZB}{(2, 4)}\textcolor{OliveGreenZB}{(1, 3)} $} (4,0);
				\end{tikzpicture} & \giToSiReductionGraphOne
			\end{tabular}
	\end{center}}
\end{frame}

\begin{frame}{Without constraints on $ \SI $ we don't have soundness}
	\textbf{\textcolor{DarkRedZB}{Problem}}: this $ \SI $ instance also has the following solution:\begin{center}
		\begin{tikzpicture}[thick]
			
			% labels
			\node (l12) {$\{1,2\}$};
			\node[right=.5em of l12] (l13) {$\{1,3\}$};
			\node[right=.5em of l13] (l14) {$\{1,4\}$};
			\node[right=.5em of l14] (l23) {$\{2,3\}$};
			\node[right=.5em of l23] (l24) {$\{2,4\}$};
			\node[right=.5em of l24] (l34) {$\{3,4\}$};
			
			% u_0
			\node[below=.5em of l12] (u0at12) {$ 1 $};
			\node[below=.5em of l13] (u0at13) {$ 1 $};
			\node[below=.5em of l14] (u0at14) {$ 0 $};
			\node[below=.5em of l23] (u0at23) {$ 1 $};
			\node[below=.5em of l24] (u0at24) {$ 0 $};
			\node[below=.5em of l34] (u0at34) {$ 1 $};
			
			\node[left=2em of u0at12] {$ u_0 $};
			
			% u_1
			\node[below=3em of u0at12] (u1at12) {$ 1 $};
			\node[below=3em of u0at13] (u1at13) {$ 1 $};
			\node[below=3em of u0at14] (u1at14) {$ 1 $};
			\node[below=3em of u0at23] (u1at23) {$ 0 $};
			\node[below=3em of u0at24] (u1at24) {$ 0 $};
			\node[below=3em of u0at34] (u1at34) {$ 1 $};
			
			\node[left=2em of u1at12] {$ u_1 $};
			
			% edges from u_0 to u_1
			\draw[->, MagentaZB] (u0at12.south) -- (u1at12.north);
			\draw[->, MagentaZB] (u0at13.south) -- (u1at13.north);
			\draw[->, MagentaZB] (u0at14.south) -- (u1at23.north);
			\draw[->, MagentaZB] (u0at23.south) -- (u1at14.north);
			\draw[->, MagentaZB] (u0at24.south) -- (u1at24.north);
			\draw[->, MagentaZB] (u0at34.south) -- (u1at34.north);
			
		\end{tikzpicture}
	\end{center}\begin{itemize}
		\item This solution does not correspond to any graph isomorphism
		\item $ \rightsquigarrow $ The reduction is complete but not sound!
	\end{itemize}

	\textbf{\textcolor{OliveGreenZB}{Solution}}: Define $ \Gamma $ to contain exactly those permutations that encode valid graph isomorphisms.

\end{frame}

%\textbf{Proof idea}: encode each graph into a string such that relabeling vertices corresponds to permuting characters of the string.
\begin{frame}{$ \GI $ reduces to $ \SI $ efficiently}
	\vspace{-4pt}\begin{theorem}
		$ \GI \le_m^p \SI $
	\end{theorem}\vspace{-4pt}
	\begin{proof}
		W.l.o.g. assume $ V_0 = V_1 = \{1, \dots, n\} $. Set $ \Sigma := \{0, 1\} $ and $ \Omega := \{\{i, j\} \mid 1 \le i < j \le n\} $. For $ b \in \{0, 1\} $ let $ u_b $ be a string defined by \[
			u_b(\{i, j\}) = 1 \iff \{i, j\} \in E_b
		\] Then \[
			G_0 \cong G_1 \iff u_0 \cong_{\ImageOf(\psi)} u_1
		\] where $ \psi : S_n \rightarrow \Sym(\Omega) $ is the embedding homomorphism \[
			\psi(\sigma) := \Big(\{i, j\} \mapsto \{\sigma(i), \sigma(j)\}\Big)
		\] Therefore, $ \ImageOf(\psi) = \gen{\psi((1, 2)), \psi((1, \dots, n))} $ and the result follows.
	\end{proof}
% This is a yes-instance to the string isomorphism problem because \[
%u_1 = u_0^{\psi((1,3)(2,4))}
%\] Indeed, $ (1,3)(2,4) \in S_4 $ is the relabeling that makes $ G_0 $ equal $ G_1 $.
\end{frame}

\newcommand{\solSpaceExGraphZero}{
	\raisebox{-.5\height}{
		\begin{tikzpicture}[every node/.style={circle, draw, fill=gray!10, inner sep=1pt, minimum size=18pt}, scale=0.8]
			\node (a3) at (0,0) {3};
			\node (a4) at (1.2,0) {4};
			\node (a2) at (1.2,1.2) {2};
			\node (a1) at (0,1.2) {1};
			\draw (a3)--(a1)--(a2)--(a3)--(a4);
		\end{tikzpicture}
	}
}

\newcommand{\solSpaceExGraphOne}{
	\raisebox{-.5\height}{
		\begin{tikzpicture}[every node/.style={circle, draw, fill=gray!10, inner sep=1pt, minimum size=18pt}, scale=0.8]
			\node (b3) at (0,0) {2};
			\node (b4) at (1.2,0) {4};
			\node (b2) at (1.2,1.2) {1};
			\node (b1) at (0,1.2) {3};
			\draw (b3)--(b1)--(b2)--(b3)--(b4);
		\end{tikzpicture}
	}
}

\newcommand{\solSpaceExGraphTwo}{
	\raisebox{-.5\height}{
		\begin{tikzpicture}[every node/.style={circle, draw, fill=gray!10, inner sep=1pt, minimum size=18pt}, scale=0.8]
			\node (b3) at (0,0) {1};
			\node (b4) at (1.2,0) {4};
			\node (b2) at (1.2,1.2) {3};
			\node (b1) at (0,1.2) {2};
			\draw (b3)--(b1)--(b2)--(b3)--(b4);
		\end{tikzpicture}
	}
}

\begin{frame}{Solutions to $ \GI $ have a very nice structure}
	\only<1-2|handout:1-2>{Consider the isomorphisms between the following graph and itself:
	\begin{center}
		\solSpaceExGraphZero
	\end{center}}
	\only<2|handout:2>{For example:\begin{center}
		\renewcommand{\arraystretch}{1.5} % row height
		\begin{tabular}{c c c}
			\solSpaceExGraphZero & \begin{tikzpicture}[thick]
				\draw[->] (0,0) --node[midway, above]{$ (1,2,3) $} (2,0);
			\end{tikzpicture} & \solSpaceExGraphOne
		\end{tabular}
	\end{center}}
	\only<3|handout:3>{But this means we can repeat the process:\begin{center}
			\renewcommand{\arraystretch}{1.5} % row height
			\begin{tabular}{c c c}
				\solSpaceExGraphZero &\hspace*{-40pt}\begin{tikzpicture}[thick]
					\draw[->] (0,0) --node[midway, above]{$ (1,2,3) $} (2,0);
				\end{tikzpicture} & \solSpaceExGraphOne\\
				\hspace*{40pt}\begin{tikzpicture}[thick]
					\draw[<-] (0,2) --node[midway, left]{$ (1,2,3) $} (2,0);
				\end{tikzpicture}&&\hspace*{-40pt}\begin{tikzpicture}[thick]
					\draw[->] (0,2) --node[midway, right]{$ (1,2,3) $} (-2,0);
				\end{tikzpicture}\\
				&\hspace*{-40pt}\solSpaceExGraphTwo&
			\end{tabular}
	\end{center}\begin{itemize}
	\item $ \rightsquigarrow $ The set of isomorphisms between a graph $ G $ and itself forms a group called automorphism group and denoted $ \Aut(G) $
	\item \textbf{This example}: $ \Aut(G) \cong S_3 $
\end{itemize}}
\end{frame}

\def\orbitGraphCoords{\coordinate (A) at (0,0);\coordinate (B) at (1,0);\coordinate (C) at (1,1);\coordinate (D) at (0,1);}

\newcommand{\orbitOne}{\begin{tikzpicture}[scale=0.25, line cap=round, line width=1pt]
		\begin{scope}[shift={(0,0)}]
			\orbitGraphCoords
			\filldraw (A) circle (1pt);
			\filldraw (B) circle (1pt);
			\filldraw (C) circle (1pt);
			\filldraw (D) circle (1pt);
		\end{scope}
\end{tikzpicture}}

\newcommand{\orbitTwo}{\begin{tikzpicture}[scale=0.25, line cap=round, line width=1pt]
		\begin{scope}[shift={(0,0)}]
			\orbitGraphCoords
			\draw (A)--(B);
		\end{scope}
		
		\begin{scope}[shift={(2,0)}]
			\orbitGraphCoords
			\draw (B)--(C);
		\end{scope}
		
		\begin{scope}[shift={(4,0)}]
			\orbitGraphCoords
			\draw (C)--(D);
		\end{scope}
		
		\begin{scope}[shift={(6,0)}]
			\orbitGraphCoords
			\draw (D)--(A);
		\end{scope}
		
		% one edge + one diagonal
		\begin{scope}[shift={(0,-2)}]
			\orbitGraphCoords
			\draw (A)--(C);
		\end{scope}
		
		\begin{scope}[shift={(2,-2)}]
			\orbitGraphCoords
			\draw (B)--(D);
		\end{scope}
\end{tikzpicture}}

\newcommand{\orbitThree}{\begin{tikzpicture}[scale=0.25, line cap=round, line width=1pt]
		% adjacent edges
		\begin{scope}[shift={(0,0)}]
			\orbitGraphCoords
			\draw (A)--(B) (B)--(C);
		\end{scope}
		
		\begin{scope}[shift={(2,0)}]
			\orbitGraphCoords
			\draw (B)--(C) (C)--(D);
		\end{scope}
		
		\begin{scope}[shift={(4,0)}]
			\orbitGraphCoords
			\draw (C)--(D) (D)--(A);
		\end{scope}
		
		\begin{scope}[shift={(6,0)}]
			\orbitGraphCoords
			\draw (D)--(A) (A)--(B);
		\end{scope}
		
		% one edge + one diagonal
		\begin{scope}[shift={(0,-2)}]
			\orbitGraphCoords
			\draw (A)--(B) (A)--(C);
		\end{scope}
		
		\begin{scope}[shift={(2,-2)}]
			\orbitGraphCoords
			\draw (A)--(B) (B)--(D);
		\end{scope}
		
		\begin{scope}[shift={(4,-2)}]
			\orbitGraphCoords
			\draw (B)--(C) (A)--(C);
		\end{scope}
		
		\begin{scope}[shift={(6,-2)}]
			\orbitGraphCoords
			\draw (B)--(C) (B)--(D);
		\end{scope}
		
		% opposite edges
		\begin{scope}[shift={(0,-4)}]
			\orbitGraphCoords
			\draw (A)--(B) (C)--(D);
		\end{scope}
		
		\begin{scope}[shift={(2,-4)}]
			\orbitGraphCoords
			\draw (B)--(C) (D)--(A);
		\end{scope}
		
		% both diagonals / cross patterns
		\begin{scope}[shift={(4,-4)}]
			\orbitGraphCoords
			\draw (A)--(C) (B)--(D);
		\end{scope}
		
		\begin{scope}[shift={(6,-4)}]
			\orbitGraphCoords
			\draw (A)--(C) (A)--(D);
		\end{scope}
\end{tikzpicture}}

\newcommand{\orbitFour}{\begin{tikzpicture}[scale=0.25, line cap=round, line width=1pt]
		% cups
		\begin{scope}[shift={(0,0)}]
			\orbitGraphCoords
			\draw (D)--(A)--(B)--(C);
		\end{scope}
		
		\begin{scope}[shift={(2,0)}]
			\orbitGraphCoords
			\draw (A)--(B)--(C)--(D);
		\end{scope}
		
		\begin{scope}[shift={(4,0)}]
			\orbitGraphCoords
			\draw (B)--(C)--(D)--(A);
		\end{scope}
		
		\begin{scope}[shift={(6,0)}]
			\orbitGraphCoords
			\draw (C)--(D)--(A)--(B);
		\end{scope}
		
		% zigzags (row 2)
		\begin{scope}[shift={(0,-2)}]
			\orbitGraphCoords
			\draw (C)--(D)--(B)--(A);
		\end{scope}
		
		\begin{scope}[shift={(2,-2)}]
			\orbitGraphCoords
			\draw (D)--(C)--(A)--(B);
		\end{scope}
		
		\begin{scope}[shift={(4,-2)}]
			\orbitGraphCoords
			\draw (A)--(D)--(B)--(C);
		\end{scope}
		
		\begin{scope}[shift={(6,-2)}]
			\orbitGraphCoords
			\draw (D)--(A)--(C)--(B);
		\end{scope}
		
		% zigzags (row 3)
		\begin{scope}[shift={(0,-4)}]
			\orbitGraphCoords
			\draw (C)--(A)--(D)--(B);
		\end{scope}
		
		\begin{scope}[shift={(2,-4)}]
			\orbitGraphCoords
			\draw (A)--(C)--(D)--(B);
		\end{scope}
		
		\begin{scope}[shift={(4,-4)}]
			\orbitGraphCoords
			\draw (D)--(B)--(C)--(A);
		\end{scope}
		
		\begin{scope}[shift={(6,-4)}]
			\orbitGraphCoords
			\draw (D)--(B)--(A)--(C);
		\end{scope}
\end{tikzpicture}}

\newcommand{\orbitFive}{\begin{tikzpicture}[scale=0.25, line cap=round, line width=1pt]
		\begin{scope}[shift={(0,0)}]
			\orbitGraphCoords
			\draw (A)--(B) (A)--(C) (A)--(D) (B)--(C) (B)--(D) (C)--(D);
		\end{scope}
\end{tikzpicture}}

\begin{frame}{Automorphism groups}
	\textbf{More generally}: $ \Aut(G) $ is isomorphic to the stabilizer in the vertex renaming operation of $ S_n $ on the set of all graphs with vertices $ \{1, \dots, n\} $.
	
	\textbf{Visualization for $ n = 4 $}:
	
	\begin{center}
		\setlength{\extrarowheight}{3pt}
		\begin{tabular}{>{\centering\arraybackslash}m{2cm} >{\centering\arraybackslash}m{2cm}}
			Orbit&Stabilizer\\
			\hline\orbitOne&$ S_4 $\\
			\hline\vspace{5pt}\orbitTwo&$ \Z_2 \times \Z_2 $\\
			\hline\vspace{5pt}\orbitThree&$ \Z_2 $\\
			\hline\vspace{5pt}\orbitFour&$ \Z_2 $\\
			\hline\vspace{5pt}\orbitFive&$ S_4 $
		\end{tabular}
	\end{center}
\end{frame}

\begin{frame}
	Let $ G $ and $ H $ be (undirected) graphs.
	\begin{itemize}
		\item Let $ \Iso(G, H) $ be the set of all isomorphisms between $ G $ and $ H $
		\item The set of isomorphisms between a graph and itself form a group \[
		\Aut(G) := \Iso(G, G)
		\] called \textit{automorphism group}.
		\item $ \Iso(G, H) $ is either empty or is a coset of the automorphism group \[
		\Iso(G, H) = \Aut(G) \sigma
		\] (for some $ \sigma \in \Iso(G, H) $).
	\end{itemize}
\end{frame}

\begin{frame}{Groups and their composition factors}
	\vspace{-5pt}\begin{definition}
		Let $ \Gamma $ be a finite group. A \textit{subnormal series} of $ \Gamma $ is a sequence \[
		\{\id\} \trianglelefteq \Gamma_1 \trianglelefteq \Gamma_2 \trianglelefteq \dots \trianglelefteq \Gamma_k = \Gamma
		\] where $ \Gamma_1, \dots, \Gamma_k $ are groups. A subnormal series is a \textit{composition series} of $ \Gamma $ if it is strictly increasing and every $ \Gamma_i/\Gamma_{i-1} $ is simple (factor groups $ \Gamma_i/\Gamma_{i-1} $ are called \textit{composition factors}).
	\end{definition}
	\vspace{-5pt}\begin{proposition}
		Every finite group can be decomposed uniquely into its composition factors.
	\end{proposition}
	\vspace{-5pt}\begin{example}
		The composition series of $ S_n $ is $ \{\id\} \trianglelefteq A_n \trianglelefteq S_n $. Thus, $ S_n $ can be decomposed as $ S_n \cong (\Z/2\Z) \times A_n $.
	\end{example}
\end{frame}

\begin{frame}{$ \widehat{\Gamma}_d $-groups}
	
	\begin{definition}[Luks]
		For $ d \ge 2 $ define $ \widehat{\Gamma}_d $ to be the class of all groups $ \Gamma $ such that every composition factor of $ \Gamma $ is isomorphic to a subgroup of $ S_d $.
	\end{definition}

	$ \widehat{\Gamma}_d $-groups are central to the $ \GI $ problem for bounded-degree graphs because:

	\begin{theorem}[Luks]
		
	\end{theorem}
\end{frame}

\end{document}